\documentclass[dvipdfmx,a4paper,11pt]{article}
\usepackage[utf8]{inputenc}
%\usepackage[dvipdfmx]{hyperref} %リンクを有効にする
\usepackage{url} %同上
\usepackage{amsmath,amssymb} %もちろん
\usepackage{amsfonts,amsthm,mathtools} %もちろん
\usepackage{braket,physics} %あると便利なやつ
\usepackage{bm} %ラプラシアンで使った
\usepackage[top=15truemm,bottom=30truemm,left=25truemm,right=25truemm]{geometry} %余白設定
\usepackage{latexsym} %ごくたまに必要になる
\renewcommand{\kanjifamilydefault}{\gtdefault}
\usepackage{otf} %宗教上の理由でmin10が嫌いなので
\usepackage{showkeys}\renewcommand*{\showkeyslabelformat}[1]{\fbox{\parbox{2cm}{ \normalfont\tiny\sffamily#1\vspace{6mm}}}}
\usepackage[driverfallback=dvipdfm]{hyperref}


\usepackage[all]{xy}
\usepackage{amsthm,amsmath,amssymb,comment}
\usepackage{amsmath}    % \UTF{00E6}\UTF{0095}°\UTF{00E5}\UTF{00AD}\UTF{00A6}\UTF{00E7}\UTF{0094}¨
\usepackage{amssymb}  
\usepackage{color}
\usepackage{amscd}
\usepackage{amsthm}  
\usepackage{wrapfig}
\usepackage{comment}	
\usepackage{graphicx}
\usepackage{setspace}
\usepackage{pxrubrica}
\usepackage{enumitem}
\usepackage{mathrsfs} 

\setstretch{1.2}


\newcommand{\R}{\mathbb{R}}
\newcommand{\Z}{\mathbb{Z}}
\newcommand{\Q}{\mathbb{Q}} 
\newcommand{\N}{\mathbb{N}}
\newcommand{\C}{\mathbb{C}} 
\newcommand{\Sin}{\text{Sin}^{-1}} 
\newcommand{\Cos}{\text{Cos}^{-1}} 
\newcommand{\Tan}{\text{Tan}^{-1}} 
\newcommand{\invsin}{\text{Sin}^{-1}} 
\newcommand{\invcos}{\text{Cos}^{-1}} 
\newcommand{\invtan}{\text{Tan}^{-1}} 
\newcommand{\Area}{\text{Area}}
\newcommand{\vol}{\text{Vol}}
\newcommand{\maru}[1]{\raise0.2ex\hbox{\textcircled{\tiny{#1}}}}
\newcommand{\sgn}{{\rm sgn}}
%\newcommand{\rank}{{\rm rank}}



   %当然のようにやる.
\allowdisplaybreaks[4]
   %もちろん.
%\title{第1回. 多変数の連続写像 (岩井雅崇, 2020/10/06)}
%\author{岩井雅崇}
%\date{2020/10/06}
%ここまで今回の記事関係ない
\usepackage{tcolorbox}
\tcbuselibrary{breakable, skins, theorems}

\theoremstyle{definition}
\newtheorem{thm}{定理}
\newtheorem{lem}[thm]{補題}
\newtheorem{prop}[thm]{命題}
\newtheorem{cor}[thm]{系}
\newtheorem{claim}[thm]{主張}
\newtheorem{dfn}[thm]{定義}
\newtheorem{rem}[thm]{注意}
\newtheorem{exa}[thm]{例}
\newtheorem{conj}[thm]{予想}
\newtheorem{prob}[thm]{問題}
\newtheorem{rema}[thm]{補足}

\DeclareMathOperator{\Ric}{Ric}
\DeclareMathOperator{\Vol}{Vol}
 \newcommand{\pdrv}[2]{\frac{\partial #1}{\partial #2}}
 \newcommand{\drv}[2]{\frac{d #1}{d#2}}
  \newcommand{\ppdrv}[3]{\frac{\partial #1}{\partial #2 \partial #3}}


%ここから本文.
\begin{document}
\pagestyle{empty}



\begin{center}
{\Large 演習問題 2024年11月19日(火)} \\

%\vspace{5pt}
%{ \large 提出締め切り 2024年月日(火) 15時10分00秒 (日本標準時刻)}
\end{center}

%\vspace{2pt}
%\begin{flushleft}
%{ \large \underline{学籍番号: \hspace{4cm} 名前  \hspace{8cm} } }
%\end{flushleft}

\begin{center}
 {\large 下の問題を解け. なお解答は配布した解答用紙に解答すること.}
  \end{center}
 ただし解答に関しては答えのみならず, 答えを導出する過程をきちんと記すこと. 
 また解答用紙は1人1枚以上提出すること.
  
  \vspace{11pt}
 
 
 問題1.  行列$A$を次で定めるとき, 以下の問いに答えよ.
 $$
 A = 
 \begin{pmatrix}
 2 &4&-3&8 \\
 3&-1&2&-5 \\
  18&0&2&12
 \end{pmatrix}
 $$
 \begin{enumerate}
   \setlength{\parskip}{0cm} % 段落間
  \setlength{\itemsep}{0cm} % 項目間
 \item $A$は何行何列の行列か?
 \item $A$の$(3,2)$成分を答えよ.
  \item $A$の第2行を答えよ.
 \item $A$の第3列を答えよ.
 \end{enumerate}
 
\vspace{5pt}
 問題2. 次の行列の計算をせよ.

 \begin{enumerate}
   \setlength{\parskip}{0cm} % 段落間
  \setlength{\itemsep}{0cm} % 項目間
\item 
$
 \begin{pmatrix} 5 & -2 \\ 0 & 1 \end{pmatrix}-  3 \begin{pmatrix} 3 & 4 \\ 1 & -2 \end{pmatrix}
$
\item $
\begin{pmatrix} 2 & 0 \\ -1 & 3 \end{pmatrix}\begin{pmatrix} 1 & 2 \\ 0 & -1 \end{pmatrix}
+  \begin{pmatrix} 3 & 1 \\ 2 & 2 \end{pmatrix}\begin{pmatrix} 1 & 0 \\ 2 & -1 \end{pmatrix}
$
\item $
\begin{pmatrix} 5 & 0 \\ 0 & 2 \end{pmatrix}^3 + 3 \begin{pmatrix} 5 & 1 \\ 1 & 2 \end{pmatrix}
$
 \end{enumerate}
 
 \vspace{5pt}
 
 問題3. 次の問いに答えよ.
 \begin{enumerate}
   \setlength{\parskip}{0cm} % 段落間
  \setlength{\itemsep}{0cm} % 項目間
\item $ \begin{pmatrix} 3 & 1 \\ 2 & 4 \end{pmatrix}$の行列式を求めよ.
\item $ \begin{pmatrix} 2 & 1 \\ -5& 3 \end{pmatrix}$の逆行列を求めよ
\item $ \begin{pmatrix} 100 & 99 \\ 99 & 100\end{pmatrix}$の逆行列を求めよ
 \end{enumerate}
 
  問題4.  
 \begin{enumerate}
   \setlength{\parskip}{0cm} % 段落間
  \setlength{\itemsep}{0cm} % 項目間
\item 
$
A=\begin{pmatrix} 4 & 2 \\ 0 & 3 \end{pmatrix}
$とする. $A$を対角化せよ. また$A^n$を$n$を用いて表せ. 
\item $
B=\begin{pmatrix} 13 & -30 \\ 5 & -12 \end{pmatrix}
$とする. $B$を対角化せよ. また$B^n$を$n$を用いて表せ. 
 \end{enumerate}
 
 \begin{flushright}
 {\Large 問題は裏面に続く}
 \end{flushright}

 \newpage
 
 問題5.   $2 \times 2$行列
$
A=\begin{pmatrix}
a& b \\
c& d \\
\end{pmatrix}
$とする. 
 $$
\begin{array}{ccccc}
f: &\R^2& \rightarrow & \R^2& \\
&\begin{pmatrix}
x \\ y
 \end{pmatrix} & \longmapsto & 
A
\begin{pmatrix}
x \\ y
 \end{pmatrix}  = 
 \begin{pmatrix}
ax + by \\ cx + dy
 \end{pmatrix}
 &
\end{array}
$$
を一次変換といい, $A$を$f$に対応する行列という. 次の問いに答えよ. 
  \begin{enumerate}
   \setlength{\parskip}{0cm} % 段落間
  \setlength{\itemsep}{0cm} % 項目間
  \item 45度反時計回りに回転する1次変換$f: \R^2 \to \R^2$に対応する$2 \times 2$行列を求めよ. 
  \item $x$軸に関しての反転(折り返し)を表す1次変換を$g : \R^2 \to \R^2$に対応する$2 \times 2$行列を求めよ.
  \item 「45度反時計回りに回転して, $x$軸に関しての反転を行う」変換は「$x$軸に関しての反転をして, 315度反時計回りに回転する」変換と同じであることを示せ. 
  
  \end{enumerate}
 
 
  問題6. 
   \begin{enumerate}
     \setlength{\parskip}{0cm} % 段落間
  \setlength{\itemsep}{0cm} 
   \item $2$次正方行列$A, B$について, $A$と$B$が正則行列ならば,  $AB$も正則行列であることを示せ. 
  \item $2$次正方行列$A, B$について, $AB$が正則行列ならば, $A$も$B$も正則行列であることを示せ.
  \end{enumerate}
 
 問題7. 
  $\bm{a}, \bm{b} \in \R^2$を
  $$|| 2\bm{a}+\bm{b}|| =|| \bm{a}+2\bm{b}|| =1 \quad\text{かつ} \quad (2\bm{a}+\bm{b})\cdot (\bm{a}+\bm{b})=\frac{1}{3}$$
 となるものとする. 
 次の問いに答えよ.
\begin{enumerate}
   \setlength{\parskip}{0cm} % 段落間
  \setlength{\itemsep}{0cm} 
\item $\bm{u}=2\bm{a}+\bm{b}$, $\bm{v}=\bm{a}+2\bm{b}$とおく. $\bm{a}+\bm{b}$を $\bm{u}$と$\bm{v}$を用いて表せ.
\item $\bm{u}, \bm{v}$は$\R^2$の正規直交基底となることを示せ. 特に基底となるので, 任意の点$\bm{p}$について, ある実数$c,d$があって, $\bm{p}=c\bm{u}+d\bm{v}$と表せられる.
\item   $$||\bm{p} - (\bm{a}+\bm{b})|| \leqq \frac{1}{3} \quad\text{かつ} \quad \bm{p} \cdot (2\bm{a}+\bm{b})\leqq \frac{1}{3}$$
となるように$\bm{p} \in \R^2$が動くとき, $||\bm{p}||$の最大値と最小値を求めよ. 
   \end{enumerate}
 
 \newpage
 
  \begin{center}
 {\Large 解答用紙}
% {\Large 演習問題の解答用紙 2024年1月11日(木) } \\
\end{center}

%\vspace{5pt}
%{ \large 提出締め切り 2024年月日(火) 15時10分00秒 (日本標準時刻)}
%\end{center}

%\vspace{2pt}
\begin{flushleft}
{ \large \underline{学籍番号: \hspace{4cm} 名前  \hspace{9cm}   }  }
\end{flushleft}
 

 \end{document}
